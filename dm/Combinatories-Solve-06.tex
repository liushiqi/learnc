\documentclass{article}

\usepackage[a4paper]{geometry}
\geometry{left=2.0cm,right=2.0cm,top=2.5cm,bottom=2.5cm}

\usepackage[UTF8]{ctex}
\usepackage{amsfonts,graphicx,amssymb,bm,amsthm}
\usepackage{algorithm,algorithmicx}
\usepackage[noend]{algpseudocode}
\usepackage{fancyhdr}
\usepackage{amsmath}
%\usepackage[fleqn]{amsmath}

\usepackage{pstricks}

\psset{xunit=.5pt,yunit=.5pt,runit=.5pt}

\newcounter{counter_exm}\setcounter{counter_exm}{1}
%\newcounter{counter_thm}\setcounter{counter_thm}{1}
%\newcounter{counter_lma}\setcounter{counter_lma}{1}
%\newcounter{counter_dft}\setcounter{counter_dft}{1}
%\newcounter{counter_clm}\setcounter{counter_clm}{1}
%\newcounter{counter_cly}\setcounter{counter_cly}{1}

\newtheorem{theorem}{{\hskip 1.7em \bf 定理}}
\newtheorem{lemma}[theorem]{\hskip 1.7em 引理}
\newtheorem{proposition}[theorem]{Proposition}
\newtheorem{claim}[theorem]{\hskip 1.7em 命题}
\newtheorem{corollary}[theorem]{\hskip 1.7em 推论}
\newtheorem{definition}[theorem]{\hskip 1.7em 定义}

\renewcommand{\emph}[1]{\begin{kaishu} #1 \end{kaishu}}

\newenvironment{solution}{{\noindent\hskip 2em \bf 解 \quad}}


\renewenvironment{proof}{{\noindent\hskip 2em \bf 证明 \quad}}{\hfill$\qed$\par\vskip1em}
\newenvironment{proofs}[1]{{\noindent\hskip 2em \bf #1 \quad}}{\hfill$\qed$\par\vskip1em}
\newenvironment{example}{{\noindent\hskip 2em \bf 例 \arabic{counter_exm}\quad}}{\addtocounter{counter_exm}{1}\par\vskip1em}

\newenvironment{concept}[1]{{\bf #1\quad} \begin{kaishu}} {\end{kaishu}\par}

\newcommand\E{\mathbb{E}}

\begin{document}

\pagestyle{fancy}
\lhead{\kaishu 中国科学院大学}
\chead{}
\rhead{\kaishu 2017年秋季学期组合数学}

\begin{center}
  {\LARGE{\bf{组合数学作业}}}\\
  {\large{\bf{第6次}}}\\
\end{center}

\begin{kaishu}
  \hfill 刘士祺 2017K8009929046
\end{kaishu}

\begin{enumerate}
  \item[1.] \textit{(2分)} 证明:正整数p是素数当且仅当
    $$(p − 1)! \equiv −1 (mod\ p)$$
    \begin{proofs}{解:}
      必要性:对于素数$p$, 存在集合$\mathbb{S} = \left\{2, 3, 4, \dots, p - 2\right\}$, 有$\forall x \in \mathbb{S}$, $\exists y \in \mathbb{S}, xy \equiv 1 (mod\ p)$。若$\exists y_1, y_2$, $xy_1 \equiv xy_2 \equiv 1 (mod\ p)$, 则$y_1 \equiv y_2 (mod p)$, 矛盾。故$\left(p - 1\right)! \equiv 1(p - 1) \equiv -1 (mod p)$ \\
      充分性:设$p = ab$, $a | (p - 1)!$, 且$b | (p - 1)!$, 故若$a \neq b$, $p | (p - 1)!$, 矛盾。否则,$p = a^2$, $gcd((p - 1)!, p) \neq 1$, 矛盾, 故$p$为素数。
    \end{proofs}

  \item[2.] \textit{(2分)} 证明:若$a, b$是正整数, $p$是素数, 则$\left(\frac{a}{p}\right)\left(\frac{b}{p}\right) = \left(\frac{ab}{p}\right)$ \\
    \begin{proofs}{解:}
      由欧拉判别法, $\left(\frac{a}{p}\right) \equiv a^{\frac{p - 1}{2}} (mod\ p)$, 则有$\left(\frac{a}{p}\right)\left(\frac{b}{p}\right) \equiv a^{\frac{p - 1}{2}} b^{\frac{p - 1}{2}} \equiv \left(ab\right)^{\frac{p - 1}{2}} \equiv \left(\frac{ab}{p}\right) (mod\ p)$, 故$\left(\frac{a}{p}\right)\left(\frac{b}{p}\right) = \left(\frac{ab}{p}\right)$
    \end{proofs}

  \item[3.] \textit{(2分)} $m(\geq n + 1)$个球放在n个盒子$B_1, B_2, \dots, B_n$当中。现在把这$m$个球拿出来,重新放入另外$n + 1$个新的盒子$B_1^∗, B_2^∗, \dots, B_{(n+1)}^*$当中,且每个新的盒子中至少有一个球。证明,存在两个球,每个都满足如下性质:其所在的新盒子比其所在的旧盒子放入的球的个数更少。 \\
    \begin{proofs}{解:}
      设$m \geq n$, 有$a^m - 1 = (a^{m - n} - 1)(a^n - 1) + (a^{m - n} - 1) + (a ^ n - 1)$, 设$gcd(a^m - 1, a^n - 1) = i$, $i | a^{m - n} - 1$, 设$gcd(a^n - 1, a^{m - n} - 1) = ki$, 故$ki | a^m - 1$, 故$gcd(a^m - 1, a^n - 1) = ki$, 故$k = 1$。 \\
      故$gcd(a^m - 1, a^n - 1) = gcd(a^n - 1, a^{m - n} - 1)$, 该式可看作对$m, n$辗转相减, 故$gcd(a^m - 1, a^n - 1) = gcd(0, a^{gcd(m, n)} - 1) = a^{gcd(m, n)}$。
    \end{proofs}

  \item[4.] \textit{(2分)} 已知 $\left\{F_n\right\}_{n = 1}^{\infty}$是 Fibonacci 数列,证明$\forall m, n \in \mathbb{N},gcd(F_m , F_n) = F_{gcd(m,n)}$。 \\
    \begin{proofs}{解:}
      由于$F_{m+n} = F_mF_{n + 1} + F_{m + 1} F{n}$, 带入$m = m - n, n = n$, 故$F_m = F_{m - n}F_{n + 1} + F{m - n + 1}F_n$, 设$gcd(F_m , F_n) = i$, $i | F_{m - n}$(由$gcd(F_{n + 1}, F_n) = 1$), 故$i = gcd(F_{m - n}, F_n)$(否则, 若$ki = gcd(F_{m - n}, F_n), k > 1$, $ki | F_m$, 矛盾。 \\
      故$gcd(F_m , F_n) = gcd(F_{m - n}, F_n)$, 同理可证, $gcd(F_m , F_n) = F_{gcd(m,n)}$。
    \end{proofs}

  \item[5.] \textit{(1分)} 证明:对于素数$p > 2, \binom{2p}{p} \equiv 2(mod\ p)$ \\
    \begin{proofs}{解:} 由卢卡斯定理
      $$\binom{2p}{p} \equiv \binom{2}{1} + \binom{0}{0} \equiv 2 (mod\ p)$$
    \end{proofs}

  \item[6.] \textit{(2分)} 对于素数$p$定义$h_p(n)$为$n!$中素数因子$p$的个数,求证$h_p(2n) \geq 2h_p(n)$。 \\
    \begin{proofs}{解:}
      由
      $$h_p(n) = \sum_{i = 1}^{\infty} \lfloor \frac{n}{ip} \rfloor$$
      和
      $$h_p(2n) = \sum_{i = 1}^{\infty} \lfloor \frac{2n}{ip} \rfloor$$
      又有设$n = pk + r, 0 \leq r < k$, 有$p = \lfloor \frac{n}{k} \rfloor$, 又$2n = 2pk + 2r$,故$\lfloor \frac{2n}{k} \rfloor = 2p$或$2p + 1 \geq 2p$。
      故
      $$\sum_{i = 1}^{\infty} \lfloor \frac{2n}{ip} \rfloor \geq 2\sum_{i = 1}^{\infty} \lfloor \frac{n}{ip} \rfloor$$
      即$h_p(2n) \geq 2h_p(n)$
    \end{proofs}

  \item[7.] \textit{(2分)} 证明:任给$m, n \in \mathbb{N}$,都有$m!n!(m + n)!|(2m)!(2n)!$。 \\
    \begin{proofs}{解:}
      设$S(m, n) = \frac{(2m)!(2n)!}{m!n!(m + n)!}$, 有
      \begin{equation*}
        \begin{aligned}
            & S(m, n + 1) + S(m + 1, n)                                                     \\
          = & S(m, n) \frac{2(2n + 1)}{(m + n + 1)} + S(m, n) \frac{2(2m + 1)}{(m + n + 1)} \\
          = & 4S(m, n)
        \end{aligned}
      \end{equation*}
      故$S(m, n + 1) = 4S(m, n) - S(m + 1, n)$, 又$S(m, 0) = \frac{(2m)!}{m!} \in \mathbb{Z}$, 故$\forall m, n, S(m, n) \in \mathbb{Z}$, 故$m!n!(m + n)!|(2m)!(2n)!$。
    \end{proofs}

  \item[8.] 设集合$A, B \in \mathbb{Z}$,定义集合$A + B = \left\{a + b|a \in A, b \in B\right\}$。证明:$|A + B| \geq |A| + |B| − 1$ \\
    \begin{proofs}{解:}
      67是素数故$(\frac{20}{67}) \equiv 20^{\frac{67 - 1}{2}} \equiv -1 (mod\ 67)$
    \end{proofs}

\end{enumerate}
\end{document}
