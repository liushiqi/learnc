\documentclass{article}

\usepackage[a4paper]{geometry}
\geometry{left=2.0cm,right=2.0cm,top=2.5cm,bottom=2.5cm}

\usepackage[UTF8]{ctex}
\usepackage{amsfonts,graphicx,amssymb,bm,amsthm}
\usepackage{algorithm,algorithmicx}
\usepackage[noend]{algpseudocode}
\usepackage{fancyhdr}
\usepackage{amsmath}
%\usepackage[fleqn]{amsmath}

\usepackage{pstricks}

\psset{xunit=.5pt,yunit=.5pt,runit=.5pt}

\newcounter{counter_exm}\setcounter{counter_exm}{1}
%\newcounter{counter_thm}\setcounter{counter_thm}{1}
%\newcounter{counter_lma}\setcounter{counter_lma}{1}
%\newcounter{counter_dft}\setcounter{counter_dft}{1}
%\newcounter{counter_clm}\setcounter{counter_clm}{1}
%\newcounter{counter_cly}\setcounter{counter_cly}{1}

\newtheorem{theorem}{{\hskip 1.7em \bf 定理}}
\newtheorem{lemma}[theorem]{\hskip 1.7em 引理}
\newtheorem{proposition}[theorem]{Proposition}
\newtheorem{claim}[theorem]{\hskip 1.7em 命题}
\newtheorem{corollary}[theorem]{\hskip 1.7em 推论}
\newtheorem{definition}[theorem]{\hskip 1.7em 定义}

\renewcommand{\emph}[1]{\begin{kaishu} #1 \end{kaishu}}

\newenvironment{solution}{{\noindent\hskip 2em \bf 解 \quad}}


\renewenvironment{proof}{{\noindent\hskip 2em \bf 证明 \quad}}{\hfill$\qed$\par\vskip1em}
\newenvironment{proofs}[1]{{\noindent\hskip 2em \bf #1 \quad}}{\hfill$\qed$\par\vskip1em}
\newenvironment{example}{{\noindent\hskip 2em \bf 例 \arabic{counter_exm}\quad}}{\addtocounter{counter_exm}{1}\par\vskip1em}

\newenvironment{concept}[1]{{\bf #1\quad} \begin{kaishu}} {\end{kaishu}\par}

\newcommand\E{\mathbb{E}}

\begin{document}

\pagestyle{fancy}
\lhead{\kaishu 中国科学院大学}
\chead{}
\rhead{\kaishu 2017年秋季学期组合数学}

\begin{center}
  {\LARGE{\bf{组合数学作业}}}\\
  {\large{\bf{第5次}}}\\
\end{center}

\begin{kaishu}
  \hfill 刘士祺 2017K8009929046
\end{kaishu}

\begin{enumerate}
  \item[1.] \textit{(2分)} 快速排序的期望时间复杂度有如下递推:
    $$T(n) = \frac{1}{n}\sum_{k = 0}^{n - 1}\left(T(k) + T(n - k - 1)\right) + O(n)$$
    求证$T(n) = O(n \cdot log(n))$。 \\
    \begin{proofs}{解:}
      由
      $$T(n) = \frac{1}{n}\sum_{k = 0}^{n - 1}\left(T(k) + T(n - k - 1)\right) + O(n)$$
      知
      $$T(n) = \frac{2}{n}\sum_{k = 0}^{n - 1}T(k) + O(n)$$
      故
      $$nT(n) = 2 \sum_{k = 0}^{n - 1}T(k) + O(n^2)$$
      用$T(n)$减去$T(n - 1)$得
      \begin{equation*}
        \begin{aligned}
          nT(n) - (n - 1)T(n - 1)                 & = 2T(n - 1) + O(n) \\
          nT(n) - (n + 1)T(n - 1)                 & = O(n)             \\
          \frac{T(n)}{n + 1} - \frac{T(n - 1)}{n} & = O(\frac{1}{n})   \\
          \frac{T(n)}{n + 1}                      & = O(log(n))        \\
          T(n)                                    & = O(n \cdot log(n) \\
        \end{aligned}
      \end{equation*}
    \end{proofs}

  \item[2.] \textit{(2分)} 证明:$\forall a, b \in \mathbb{N}, \exists p, q \in \mathbb{Z}, ap + bq = gcd(a, b)$。 \\
    \begin{proofs}{解:}
      设$d$为集合$\mathbb{S} = \left\{ax + by | x, y \in\mathbb{Z}, ax + by \geq 0 \right\}$中的最小值$ap + bq$。设$a = dx + r, 0 \leq r < a$, 由$r = a - xd = a - x(ap + bq) = a(1 - xp) - bxq$, 又$r \in \mathbb{S}$, $d$为$\mathbb{S}$中最小正数,故$r = 0$, 故$d | a$, 同理, $d | b$。 \\
      对$\forall i, i | a \& i | b$, 设$a = mi, b = ni$, $d = ap + bq = i(mp + nq)$, 故$i | k$, 故$k = gcd(a, b)$。
    \end{proofs}

  \item[3.] \textit{(2分)} 证明$\forall a > 1, m, n \in \mathbb{N}, gcd\left(a^m − 1, a^n − 1\right) = a^{gcd(m,n)} − 1$。 \\
    \begin{proofs}{解:}
      设$m \geq n$, 有$a^m - 1 = (a^{m - n} - 1)(a^n - 1) + (a^{m - n} - 1) + (a ^ n - 1)$, 设$gcd(a^m - 1, a^n - 1) = i$, $i | a^{m - n} - 1$, 设$gcd(a^n - 1, a^{m - n} - 1) = ki$, 故$ki | a^m - 1$, 故$gcd(a^m - 1, a^n - 1) = ki$, 故$k = 1$。 \\
      故$gcd(a^m - 1, a^n - 1) = gcd(a^n - 1, a^{m - n} - 1)$, 该式可看作对$m, n$辗转相减, 故$gcd(a^m - 1, a^n - 1) = gcd(0, a^{gcd(m, n)} - 1) = a^{gcd(m, n)}$。
    \end{proofs}

  \item[4.] \textit{(2分)} 已知 $\left\{F_n\right\}_{n = 1}^{\infty}$是 Fibonacci 数列,证明$\forall m, n \in \mathbb{N},gcd(F_m , F_n) = F_{gcd(m,n)}$。 \\
    \begin{proofs}{解:}
      由于$F_{m+n} = F_mF_{n + 1} + F_{m + 1} F{n}$, 带入$m = m - n, n = n$, 故$F_m = F_{m - n}F_{n + 1} + F{m - n + 1}F_n$, 设$gcd(F_m , F_n) = i$, $i | F_{m - n}$(由$gcd(F_{n + 1}, F_n) = 1$), 故$i = gcd(F_{m - n}, F_n)$(否则, 若$ki = gcd(F_{m - n}, F_n), k > 1$, $ki | F_m$, 矛盾。 \\
      故$gcd(F_m , F_n) = gcd(F_{m - n}, F_n)$, 同理可证, $gcd(F_m , F_n) = F_{gcd(m,n)}$。
    \end{proofs}

  \item[5.] \textit{(1分)} 证明:对于素数$p > 2, \binom{2p}{p} \equiv 2(mod\ p)$ \\
    \begin{proofs}{解:} 由卢卡斯定理
      $$\binom{2p}{p} \equiv \binom{2}{1} + \binom{0}{0} \equiv 2 (mod\ p)$$
    \end{proofs}

  \item[6.] \textit{(2分)} 对于素数$p$定义$h_p(n)$为$n!$中素数因子$p$的个数,求证$h_p(2n) \geq 2h_p(n)$。 \\
    \begin{proofs}{解:}
      由
      $$h_p(n) = \sum_{i = 1}^{\infty} \lfloor \frac{n}{ip} \rfloor$$
      和
      $$h_p(2n) = \sum_{i = 1}^{\infty} \lfloor \frac{2n}{ip} \rfloor$$
      又有设$n = pk + r, 0 \leq r < k$, 有$p = \lfloor \frac{n}{k} \rfloor$, 又$2n = 2pk + 2r$,故$\lfloor \frac{2n}{k} \rfloor = 2p$或$2p + 1 \geq 2p$。
      故
      $$\sum_{i = 1}^{\infty} \lfloor \frac{2n}{ip} \rfloor \geq 2\sum_{i = 1}^{\infty} \lfloor \frac{n}{ip} \rfloor$$
      即$h_p(2n) \geq 2h_p(n)$
    \end{proofs}

  \item[7.] \textit{(2分)} 证明:任给$m, n \in \mathbb{N}$,都有$m!n!(m + n)!|(2m)!(2n)!$。 \\
    \begin{proofs}{解:}
      设$S(m, n) = \frac{(2m)!(2n)!}{m!n!(m + n)!}$, 有
      \begin{equation*}
        \begin{aligned}
            & S(m, n + 1) + S(m + 1, n)                                                     \\
          = & S(m, n) \frac{2(2n + 1)}{(m + n + 1)} + S(m, n) \frac{2(2m + 1)}{(m + n + 1)} \\
          = & 4S(m, n)
        \end{aligned}
      \end{equation*}
      故$S(m, n + 1) = 4S(m, n) - S(m + 1, n)$, 又$S(m, 0) = \frac{(2m)!}{m!} \in \mathbb{Z}$, 故$\forall m, n, S(m, n) \in \mathbb{Z}$, 故$m!n!(m + n)!|(2m)!(2n)!$。
    \end{proofs}

  \item[8.] 计算下列式子,其中$(\frac{a}{p})$表示 Legendre 符号,即如果$a$是$p$的二次剩余, 则$(\frac{a}{p}) = 1$,如果$a$是$p$的二次非剩余,则$(\frac{a}{p}) = −1$ \\
    \begin{enumerate}
      \item[a)] \textit{(1分)} $(\frac{20}{67})$ \\
        \begin{proofs}{解:}
          67是素数故$(\frac{20}{67}) \equiv 20^{\frac{67 - 1}{2}} \equiv -1 (mod\ 67)$
        \end{proofs}
      \item[b)] \textit{(1分)} $(\frac{14}{73})$ \\
        \begin{proofs}{解:}
          73是素数故$(\frac{14}{73}) \equiv 14^{\frac{73 - 1}{2}} \equiv -1 (mod\ 73)$
        \end{proofs}
    \end{enumerate}

  \item[9.] \textit{(2分)} $p = 6k + 5(k \in \mathbb{N})$是素数,计算 $(\frac{−3}{p})$。 \\
    \begin{proofs}{解}

    \end{proofs}

  \item[10.] 将$1 \sim 2n$填入$2 \times n$的杨氏图表(即要求图表中每行每列均单调递增),有多少种不同的方案? \\
    \begin{proofs}{解:}
      设有$s(n)$种方案, 左上角必为1, 右下角必为$2n$, 若$2n$的上面为$2n - 1$, 则左面为$2n - 2$, 去掉最后两个, 方法数共为$s(n - 1)$, 若左面为$2n - 1$, 设上面为$b(n \leq b < 2n - 1)$, 将$b + 1$换为$b$, $b + 2$换为$b + 1$ \dots,$b$换为$n - 1$, 化为前一种情况。故$s(n) = s(n - 1) + (n - 1)s(n - 1) = n s(n - 1)$又$s(1) = 1$故$s(n) = n!$。
    \end{proofs}

\end{enumerate}
\end{document}
